\documentclass[a4paper,12pt]{article}
\usepackage[utf8]{inputenc}
\usepackage[T1]{fontenc}
\usepackage{hyperref}

\title{Instructions to Run the Project}
\author{}
\date{}

\begin{document}

\maketitle

\section*{Prerequisites}
\begin{enumerate}
    \item \textbf{Java Development Kit (JDK):} Ensure you have Java 8 or a newer version installed on your system.
    \item \textbf{Docker Installed:} Ensure Docker is installed and running on your system.
    \item \textbf{SMTP Server (via Docker):} The SMTP server will be launched using Docker. Ensure Docker is configured correctly.
\end{enumerate}

\section*{Steps to Launch the Project}

\subsection*{1. Clone or Download the Project}
Clone the repository or download the project files to your local machine.

\subsection*{2. Start the SMTP Server}
Launch the SMTP server using Docker with the following command:
\begin{verbatim}
docker run -d --name smtp-server -p 1025:1025 -p 8025:8025 mailhog/mailhog
\end{verbatim}
This will start a MailHog SMTP server:
\begin{itemize}
    \item \textbf{SMTP port:} 1025
    \item \textbf{Web interface:} \url{http://localhost:8025}
\end{itemize}
Verify that the SMTP server is running by visiting the web interface in your browser.

\subsection*{3. Configure the Project}
The configuration is stored in the JSON file located at:
\begin{verbatim}
lab-smtp/resources/config.json
\end{verbatim}
To adjust specific parameters:
\begin{itemize}
    \item \textbf{Group Size:} Modify the \texttt{groupSize} value (must be between 2 and 5).
    \item \textbf{SMTP Server IP Address and Port:} Update the \texttt{ipAddress} to \texttt{127.0.0.1} and the \texttt{port} to \texttt{1025}.
\end{itemize}

\subsection*{4. Modify JSON Files for Customization}
The project uses the following JSON files for input:
\begin{itemize}
    \item \texttt{victims.json} (located in \texttt{lab-smtp/resources/}) to specify victim email addresses.
    \item \texttt{mailMsgs.json} (located in \texttt{lab-smtp/resources/}) to define email subjects and bodies.
\end{itemize}
To customize:
\begin{enumerate}
    \item Open the respective JSON files in a text editor.
    \item Add, modify, or remove entries as needed, following the existing format.
\end{enumerate}

\subsection*{5. Compile the Project}
Navigate to the project's root directory and compile the project using the following command:
\begin{verbatim}
javac -d bin src/dai/lab/smtp/*.java
\end{verbatim}
This will compile all Java files and place the output in the \texttt{bin} directory.

\subsection*{6. Run the Application}
Execute the main application from the \texttt{bin} directory:
\begin{verbatim}
java dai.lab.smtp.App
\end{verbatim}

\subsection*{7. Monitor the Output}
The application will load the configuration, divide victims into groups, and send prank emails. 
Console logs will display the progress, including information about the groups and emails being sent. 
Open the MailHog web interface (\url{http://localhost:8025}) to view the sent emails.

\section*{Troubleshooting}
\begin{itemize}
    \item \textbf{Invalid \texttt{groupSize}:} Ensure \texttt{groupSize} in \texttt{config.json} is set between 2 and 5. An error will occur if this condition is not met.
    \item \textbf{Docker Issues:} Verify that Docker is installed and running. Ensure the MailHog container is started using the correct command.
    \item \textbf{JSON Parsing Errors:} Ensure the syntax of \texttt{victims.json} and \texttt{mailMsgs.json} is correct. Use a JSON validator if needed.
\end{itemize}

\noindent Feel free to modify the JSON files in the \texttt{resources} folder to update the group size, victim emails, or email messages as needed.

\end{document}
